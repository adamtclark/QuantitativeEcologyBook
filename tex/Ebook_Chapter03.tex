\title{Chapter 3 Practice Questions}

\documentclass[12pt]{article}
\usepackage{amsmath}

\begin{document}
\date{}
\maketitle
In the text, we presented the following example to simulate the growth of a population that doubles in size every hour:

\begin{verbatim}
N=1; t=0;
while(t<=5*24) { print(N); N=N*2; t=t+1; }
\end{verbatim}

\noindent{\textit{Question 3.1:}}
\newline
How would we update this code to simulate growth of a population that triples in size every hour?
\newline

\noindent{\textit{Question 3.2:}}
\newline
How would we update this code to simulate growth of a population that doubles only every ten hours?
\newline

\noindent{\textit{Question 3.3:}}
\newline
How would we update this code to show us how large the population will be after three days of growth?

\pagebreak
\noindent{\textbf{Answers:}}

~\newline
\textit{Question 3.1:}
\newline
Instead of multiplying \verb!N! by 2 every time-step, we would instead multiply it by 3, yielding:

\begin{verbatim}
N=1; t=0;
while(t<=5*24) { print(N); N=N*3; t=t+1; }
\end{verbatim}

~\\~\newline
\textit{Question 3.2:}
\newline
Instead of moving forward in time-steps of one hour (i.e. \verb!t=t+1!), we would use time-steps of ten hours, yielding:

\begin{verbatim}
N=1; t=0;
while(t<=5*24) { print(N); N=N*2; t=t+10; }
\end{verbatim}

~\\~\newline
\textit{Question 3.3:}
\newline
In order to simulate only 3 days of growth, rather than 5, we would update the total number of iterations in \verb!t<5*24! to \verb!t<3*24!, yielding:

\begin{verbatim}
N=1; t=0;
while(t<=3*24) { print(N); N=N*2; t=t+1; }
\end{verbatim}

\end{document}